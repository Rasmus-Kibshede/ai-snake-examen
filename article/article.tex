\documentclass[a4paper, twocolumn]{article}
 
 
% you can switch between these two (and more) styles by commenting one out (use percentage)
\usepackage[backend=biber]{biblatex}
%\usepackage[backend=biber, style=authoryear-icomp]{biblatex}
\addbibresource{./refs.bib}
 
\usepackage{graphicx}
 
\usepackage{listings}
\usepackage{color}
\definecolor{lightgray}{gray}{0.9}
 
% code listing: https://tex.stackexchange.com/questions/19004/how-to-format-an-inline-source-code
\lstset{
    showstringspaces=false,
    basicstyle=\ttfamily,
    keywordstyle={blue},
    commentstyle=\color[gray]{0.6}
    stringstyle=\color[RGB]{255, 150, 75}
}
\newcommand{\inlinecode}[2]{\colorbox{lightgray}{\lstinline[language=#1]$#2$}}
 
\author{Jeffrey Roed, Rasmus Kibshede, Joachim Richter}
\title{Genetic algorithms snake game}
 
 
\begin{document}
 
\twocolumn[
    \begin{@twocolumnfalse}
        \maketitle
        \begin{abstract}
            ---------------- TBA --------------------
        \end{abstract}
    \end{@twocolumnfalse}
    \vspace{1cm}
]
 
 
\section{Introduction\label{sec:Introduction}}
 
The primary objective of this study is to demonstrate how genetic algorithms (GAs) can be utilized to develop an agent that plays the Snake game. The Snake game, a classic arcade game, involves navigating a snake to collect food items while avoiding collisions with walls and the snake's own body. The GA is used to optimize the snake's behavior over successive generations, improving its ability to survive and collect food.
 
\subsection{Research question\label{sec:Research Question}}
 
How can we develop a genetic algorithm capable of outperforming the average score achieved by our group in the game of Snake?
 
 
\section{Method\label{sec:Method}}
 
\subsection{Selection\label{sec:Selection}}
 
\begin{itemize}
    \item A selection mechanism is used to choose parent chromosomes for the next generation.
    \item Techniques such as roulette wheel selection or tournament selection are applied to favor higher fitness individuals.
\end{itemize}
 
\subsection{Crossover\label{sec:Crossover}}
 
\begin{itemize}
    \item Crossover operators combine pairs of parent chromosomes to produce offspring.
    \item Single-point or two-point crossover methods are employed to mix the neural network weights and biases of parent agents.
\end{itemize}
 
\subsection{Mutation\label{sec:Mutation}}
 
\begin{itemize}
    \item Mutation operators introduce variability by randomly altering parts of an offspring's chromosome.
    \item A small mutation rate is used to ensure genetic diversity and avoid premature convergence.
\end{itemize}
 
 
\section{Analysis\label{sec:Analysis}}
 
\subsection{Selection\label{sec:Selection}}
 
\subsection{Crossover\label{sec:Crossover}}
 
\subsection{Mutation\label{sec:Mutation}}
 
 
\section{Findings\label{sec:Findings}}
 
\subsection{Selection Enhancement\label{sec:Selection Enhancement}}
 
\subsection{Crossover Enhancement\label{sec:Crossover Enhancement}}
 
\subsection{Mutation Enhancement\label{sec:Mutation Enhancement}}
 
\printbibliography
 
\end{document}